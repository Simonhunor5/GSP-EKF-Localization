\section{Következtetések}

\subsection{Összefoglalás}
Ebben a munkában három különböző Extended Kalman Filter (EKF) alapú lokalizációs módszert hasonlítottunk össze szimulált range-only mérések alapján, outlier-terhelt környezetben \cite{Chai2020}. A három vizsgált módszer:
\begin{enumerate}
    \item \textbf{(A) Baseline EKF} fix mérési kovarianciával a (\ref{eq:R_baseline}) szerint \cite{Kalman1960, Grewal2001},
    \item \textbf{(B) Adaptív EKF} heurisztikus, reziduál-alapú súlyozással a (\ref{eq:heuristic_weight})--(\ref{eq:adaptive_R}) alapján \cite{Chhabra2021},
    \item \textbf{(C) GSP-alapú EKF} Laplacián tanulással a (\ref{eq:laplacian_learning}) és low-pass reziduál simítással a (\ref{eq:lowpass_smoothing}) szerint \cite{Dong2019, Shuman2013}.
\end{enumerate}

A szimulációs eredmények egyértelműen igazolják, hogy az adaptív módszerek jelentős javulást érnek el a baseline-hoz képest. A (B) módszer 7.9\%-kal, míg a (C) módszer 11.9\%-kal csökkentette az RMSE-t a (\ref{eq:rmse}) metrika szerint.

\subsection{Főbb megállapítások}
A kutatás során a következő főbb megállapításokat tettük:

\begin{itemize}
    \item \textbf{A heurisztikus súlyozás hatékony.} A (B) módszer egyszerű és számításilag olcsó megoldást kínál az outlierek kezelésére \cite{Chhabra2021}. Az anchoronkénti súlyok a (\ref{eq:heuristic_weight}) alapján dinamikusan alkalmazkodnak a mérések megbízhatóságához.
    
    \item \textbf{A GSP-alapú megközelítés további javulást hoz.} A (C) módszer a reziduálokat gráfjelként értelmezi, és a tanult Laplacián mátrixon keresztül simítja azokat a (\ref{eq:lowpass_smoothing}) szerint \cite{Dong2019}. Ez különösen akkor előnyös, ha a mérési hibák korreláltak vagy strukturáltak \cite{Shuman2013, Ortega2018GSP}.
    
    \item \textbf{A módszerek kombinálhatók.} A heurisztikus és a GSP-alapú megközelítés nem zárja ki egymást -- egy hibrid rendszer potenciálisan mindkét előnyt egyesítheti.
    
    \item \textbf{A paraméterek hangolása fontos.} Mind a $\beta$ paraméter a (\ref{eq:heuristic_weight})-ben, mind a $\gamma$ paraméter a (\ref{eq:lowpass_smoothing})-ben jelentősen befolyásolja a teljesítményt. A túl agresszív hangolás instabilitáshoz vezethet \cite{Grewal2001}.
\end{itemize}

\subsection{A munka tudományos hozzájárulása}
A bemutatott kutatás három területen járul hozzá a szakirodalomhoz:
\begin{enumerate}
    \item Demonstráltuk a GSP-alapú Laplacián tanulás alkalmazhatóságát EKF mérési zajkezelésre, ami a \cite{Dong2019, Dong2020GSPML} által javasolt módszer új alkalmazási területe.
    \item Összehasonlító elemzést adtunk a heurisztikus \cite{Chhabra2021} és gráf-alapú adaptív módszerek között.
    \item Nyílt forráskódú Python implementációt készítettünk, amely reprodukálható kísérleteket tesz lehetővé.
\end{enumerate}

\subsection{Korlátok}
A bemutatott eredmények bizonyos korlátokkal rendelkeznek:
\begin{itemize}[noitemsep]
    \item A szimulációs környezet egyszerűsített: 2D mozgás a (\ref{eq:kinematics}) kinematika szerint, konstans sebesség, stacionárius anchorok.
    \item Az outlierek modellje a (\ref{eq:outlier_model}) Gauss-keverék, ami nem feltétlenül reprezentálja a valós multipath vagy NLOS hibákat \cite{Chai2020}.
    \item A (C) módszer számításigénye nagyobb a (\ref{eq:laplacian_learning}) konvex optimalizáció miatt.
    \item A paraméterek (ablakméret, $\beta$, $\gamma$) hangolása manuális volt.
\end{itemize}

\subsection{Jövőbeli kutatási irányok}
A munka folytatásaként a következő irányok ígéretesek:
\begin{itemize}
    \item \textbf{Valós mérések:} A módszerek tesztelése UWB vagy WiFi RTT alapú valós adatokon.
    \item \textbf{3D kiterjesztés:} A (\ref{eq:state}) modell és a gráfstruktúra kiterjesztése háromdimenziós lokalizációra.
    \item \textbf{Online Laplacián tanulás:} A gráf folyamatos frissítése új mérések alapján, csökkentett számításigénnyel \cite{Dong2020GSPML}.
    \item \textbf{GCN integráció:} A tanult gráf felhasználása Graph Convolutional Network (GCN) alapú predikciós modulban \cite{Kipf2017, Wu2021}.
    \item \textbf{Szenzorfúzió:} IMU és egyéb szenzorok integrálása a (\ref{eq:measurement_model}) mérési modellbe.
    \item \textbf{Automatikus paraméterhangolás:} Bayesiánus optimalizáció vagy cross-validation alkalmazása a hiperparaméterekre.
\end{itemize}

\subsection{Záró gondolatok}
A GSP-alapú módszerek ígéretes eszközt kínálnak a szenzoradat-feldolgozáshoz \cite{Ortega2022, Sandryhaila2014}, különösen olyan helyzetekben, ahol a mérések közötti kapcsolatok (gráfstruktúra) releváns információt hordoznak. Az EKF és a gráfjelfeldolgozás kombinálása új lehetőségeket nyit a robusztus lokalizáció területén. A bemutatott eredmények alapján a (C) típusú, Laplacián-simításon alapuló megközelítés további kutatásra érdemes, különösen valós alkalmazások felé történő kiterjesztés esetén.
