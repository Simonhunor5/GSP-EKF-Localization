\section{EKF alapú lokalizáció}

\subsection{Az EKF szerepe és alapfeltevései}
Az \textit{Extended Kalman Filter} (EKF) nemlineáris állapot--tér modellek esetén ad rekurzív becslést a robot állapotára \cite{Kalman1960, Grewal2001}.
A módszer lokális linearizálást alkalmaz a mozgás- és mérési modell környezetében, majd a klasszikus Kalman-szűrő lépéseit hajtja végre.

A (\ref{eq:motion_model}) és (\ref{eq:measurement_model}) egyenletekből kiindulva:
\begin{equation}
\label{eq:ekf_system}
x_{k+1} = f(x_k,u_k) + w_k,\qquad w_k\sim\mathcal{N}(0,Q),
\end{equation}
\begin{equation}
\label{eq:ekf_measurement}
z_k = h(x_k) + v_k,\qquad v_k\sim\mathcal{N}(0,R).
\end{equation}
A baseline esetben a mérési kovariancia $R$ rögzített (időben állandó), és az EKF a Gauss zajfeltevésre épít.

\subsection{Predikciós lépés}
A predikció során az állapotot és a kovarianciát előre propagáljuk:
\begin{equation}
\label{eq:ekf_predict_state}
\hat x^-_{k} = f(\hat x_{k-1},u_{k-1}),
\end{equation}
\begin{equation}
\label{eq:ekf_predict_cov}
P^-_{k} = F_{k-1}\,P_{k-1}\,F_{k-1}^\top + Q,
\end{equation}
ahol $F_{k-1}$ a mozgásmodell Jacobianja a (\ref{eq:jacobian_F}) szerint.

\subsection{Frissítési lépés}
A frissítés során először kiszámítjuk a predikált mérést, majd ennek és a valós mérésnek a különbségét képezzük. Ezt a különbséget \textit{reziduálnak} nevezzük:
\begin{equation}
\label{eq:predicted_measurement}
\hat z_k = h(\hat x^-_k),
\end{equation}
\begin{equation}
\label{eq:residual}
r_k = z_k - \hat z_k.
\end{equation}
A reziduál azt fejezi ki, mennyire tér el a tényleges mérés attól, amit a modell alapján vártunk.

A linearizált mérési modell Jacobianja a (\ref{eq:jacobian_H_def}) alapján:
\begin{equation}
\label{eq:ekf_H}
H_k = \left.\frac{\partial h(x)}{\partial x}\right|_{x=\hat x^-_k}.
\end{equation}
A reziduál kovarianciája:
\begin{equation}
\label{eq:residual_cov}
S_k = H_k\,P^-_k\,H_k^\top + R.
\end{equation}
A Kalman-erősítés:
\begin{equation}
\label{eq:kalman_gain}
K_k = P^-_k\,H_k^\top\,S_k^{-1}.
\end{equation}
Az állapot és a kovariancia frissítése:
\begin{equation}
\label{eq:ekf_update_state}
\hat x_k = \hat x^-_k + K_k\,r_k,
\end{equation}
\begin{equation}
\label{eq:ekf_update_cov}
P_k = (I - K_k H_k)\,P^-_k.
\end{equation}
Megjegyzés: a yaw szög esetén minden iterációban szög-normalizálást végzünk, hogy $\psi\in(-\pi,\pi]$ maradjon.

\subsection{Baseline definíció: (A) EKF fix mérési kovarianciával}
A (A) pipeline a fenti EKF-et változtatás nélkül használja, rögzített mérési zajmodellel:
\begin{equation}
\label{eq:R_baseline}
R = \sigma_r^2 I_M,
\end{equation}
ahol $I_M$ az $M\times M$ egységmátrix, $\sigma_r$ pedig a range mérés szórása.
Ez a referencia megoldás lesz az adaptív módszerek (B) és (C) összehasonlításához.

\subsection{Értékelési metrika: pozíció RMSE}
A teljesítményt a pozíció becslési hibája alapján értékeljük.
Jelölje a valódi pozíciót $p(k)=[p_x(k),p_y(k)]^\top$, a becsült pozíciót pedig $\hat p(k)$.
Az időpillanat szerinti euklideszi hiba:
\begin{equation}
\label{eq:position_error}
e(k) = \| \hat p(k) - p(k) \|_2.
\end{equation}
Az $T$ hosszú szimulációra számolt RMSE:
\begin{equation}
\label{eq:rmse}
\text{RMSE} = \sqrt{\frac{1}{T}\sum_{k=1}^{T} e(k)^2 }.
\end{equation}
A későbbi fejezetekben mindhárom pipeline (A/B/C) esetén ezt a metrikát használjuk, valamint a hibagörbéket $e(k)$ is vizsgáljuk.

\subsection{Megjegyzés az outlierek hatásáról a baseline esetben}
A (\ref{eq:R_baseline}) fix $R$ választás azt jelenti, hogy az EKF minden anchor mérését azonos megbízhatóságúnak tekinti.
Outlier esetén ez túl optimista: a hibás mérés nagy reziduált generál a (\ref{eq:residual}) szerint, és aránytalanul nagy korrekciót okoz a (\ref{eq:ekf_update_state}) állapotfrissítésben.
Ez a jelenség motiválja a (B) pipeline heurisztikus, reziduál-alapú adaptív súlyozását, valamint a (C) pipeline gráf-alapú reziduál simítását.
