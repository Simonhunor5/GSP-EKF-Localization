\section{Heurisztikus mérési súlyozás reziduál alapján}

\subsection{Kiindulás: reziduál anchoronként}
A (\ref{eq:residual}) egyenletben definiált reziduál a predikált és a valós mérés különbsége \cite{Grewal2001}. Mivel $z_k\in\mathbb{R}^M$ tartalmazza az összes anchorhoz tartozó range mérést, ezért a reziduál is felbontható anchoronként:
\begin{equation}
\label{eq:residual_vector}
r_k =
\begin{bmatrix}
r_1(k)\\
\vdots\\
r_M(k)
\end{bmatrix}.
\end{equation}
Intuitívan: ha egy anchor mérése megbízhatatlan (pl. outlier vagy NLOS), akkor az adott $r_i(k)$ jellemzően nagyobb abszolút értékeket vesz fel, azaz a mért és várt távolság jobban eltér egymástól.

\subsection{Ablakolt reziduál-energia: $e_i(k)$}
A (B) pipeline célja, hogy minden anchorhoz egy online becsült megbízhatósági mérőszámot rendeljen \cite{Chhabra2021}.
Ehhez az $i$-edik anchor reziduáljából egy csúszóablakos energiát számolunk:
\begin{equation}
\label{eq:residual_energy}
e_i(k) = \frac{1}{T_{\text{win}}}\sum_{\tau=k-T_{\text{win}}+1}^{k} r_i(\tau)^2,
\end{equation}
ahol $T_{\text{win}}$ az ablak hossza.
Kis $e_i(k)$ azt jelzi, hogy az anchor mérése az utóbbi időben következetesen illeszkedik a modellhez; nagy $e_i(k)$ pedig gyakori vagy erős eltérésekre utal.

\subsection{Heurisztikus súlyok: $w_i(k)$}
A (\ref{eq:residual_energy}) energia alapján egy $[0,1]$ intervallumba eső súlyt definiálunk:
\begin{equation}
\label{eq:heuristic_weight}
w_i(k) = \exp\bigl(-\beta\,e_i(k)\bigr),
\end{equation}
ahol $\beta>0$ egy hangolási paraméter.
Ennek hatása:
\begin{itemize}
    \item ha $e_i(k)\approx 0$, akkor $w_i(k)\approx 1$ (magas megbízhatóság),
    \item ha $e_i(k)$ nő, akkor $w_i(k)$ gyorsan csökken (alacsony megbízhatóság).
\end{itemize}
A gyakorlatban numerikai okokból a súlyokat levágjuk egy kis alsó korláttal:
\begin{equation}
\label{eq:weight_clipping}
w_i(k)\leftarrow \max(w_i(k), w_{\min}),
\end{equation}
hogy az illesztés stabil maradjon.

\subsection{Adaptív mérési kovariancia: $R(k)$}
A (\ref{eq:residual_cov}) egyenletben látható, hogy a mérési kovariancia $R$ szabja meg, mennyire bízik a filter a mérésben.
A (\ref{eq:heuristic_weight}) súlyokat úgy építjük be, hogy az $i$-edik mérés effektív szórása nőjön, ha az anchor megbízhatatlan:
\begin{equation}
\label{eq:adaptive_R_element}
R_i(k) = \frac{\sigma_r^2}{w_i(k)+\varepsilon},
\end{equation}
ahol $\varepsilon>0$ kis konstans a numerikai stabilitásért.
Így:
\begin{itemize}
    \item ha $w_i(k)\approx 1$, akkor $R_i(k)\approx \sigma_r^2$ (a (\ref{eq:R_baseline}) baseline érték),
    \item ha $w_i(k)\ll 1$, akkor $R_i(k)$ megnő, tehát az EKF kevésbé támaszkodik az adott mérésre.
\end{itemize}
A teljes mérési kovariancia diagonális mátrix:
\begin{equation}
\label{eq:adaptive_R}
R(k) = \mathrm{diag}\bigl(R_1(k),\dots,R_M(k)\bigr).
\end{equation}

\subsection{(B) Pipeline összefoglalása}
A (B) módszer lépései egy időpillanatban:
\begin{enumerate}
    \item EKF predikció a (\ref{eq:ekf_predict_state})--(\ref{eq:ekf_predict_cov}) szerint: $\hat x^-_k, P^-_k$.
    \item Predikált mérés a (\ref{eq:predicted_measurement}) szerint: $\hat z_k = h(\hat x^-_k)$, reziduál: $r_k=z_k-\hat z_k$.
    \item Reziduál-ablak frissítése, majd $e_i(k)$ számítása a (\ref{eq:residual_energy}) és $w_i(k)$ a (\ref{eq:heuristic_weight}) alapján.
    \item Adaptív $R(k)$ felépítése a (\ref{eq:adaptive_R}) szerint.
    \item EKF frissítés a (\ref{eq:ekf_update_state})--(\ref{eq:ekf_update_cov}) alapján az adaptív $R(k)$ használatával.
\end{enumerate}

\subsection{Értelmezés és várható hatás}
A heurisztika erőssége, hogy egyszerű, gyors, és nem igényel gráftanulást vagy optimalizációt.
Outlier esetén az érintett anchor reziduál-energiája megemelkedik a (\ref{eq:residual_energy}) szerint, a súlya csökken a (\ref{eq:heuristic_weight}) alapján, így az EKF automatikusan csökkenti az adott mérés hatását.
Ez általában csökkenti a hirtelen hibaugrásokat és javítja az átlagos pozícióbecslési hibát a (\ref{eq:rmse}) metrika szerint.

A módszer korlátja, hogy az anchorok közötti kapcsolatot nem modellezi: minden mérést egymástól függetlenül súlyoz.
Ezt a hiányosságot célozza a (C) pipeline, amely a reziduálokat gráfon értelmezi és gráf-alapú simítást alkalmaz.
