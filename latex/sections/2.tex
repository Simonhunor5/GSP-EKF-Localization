\section{Rendszermodell és jelölések}

\subsection{Állapotvektor és bemenetek}
A robot állapotát a következő vektor írja le \cite{Grewal2001}:
\begin{equation}
\label{eq:state}
x_k =
\begin{bmatrix}
p_x(k)\\
p_y(k)\\
\psi(k)
\end{bmatrix},
\end{equation}
ahol $p_x(k)$ és $p_y(k)$ a robot pozíciója a síkban, míg $\psi(k)$ a yaw (irány-) szög.
A bemeneti vektor a sebesség és a szögsebesség:
\begin{equation}
\label{eq:input}
u_k =
\begin{bmatrix}
v(k)\\
\omega(k)
\end{bmatrix}.
\end{equation}
A diszkrét idejű mintavételi idő $dt$.

\subsection{Mozgásmodell (diszkrét idejű kinematika)}
A robot mozgását egy egyszerű kinematikai modell írja le. A diszkrét állapotegyenlet:
\begin{equation}
\label{eq:motion_model}
x_{k+1} = f(x_k,u_k) + w_k,
\end{equation}
ahol $w_k$ a folyamatzaj (process noise). A használt determinisztikus modell:
\begin{equation}
\label{eq:kinematics}
\begin{aligned}
p_x(k+1) &= p_x(k) + v(k)\,dt\,\cos(\psi(k)),\\
p_y(k+1) &= p_y(k) + v(k)\,dt\,\sin(\psi(k)),\\
\psi(k+1) &= \psi(k) + \omega(k)\,dt,
\end{aligned}
\end{equation}
ahol a $\psi$ szög értékét minden lépésben $(-\pi,\pi]$ intervallumba normáljuk.

A folyamatzaj kovarianciáját $Q$ jelöli:
\begin{equation}
\label{eq:process_noise}
w_k \sim \mathcal{N}(0,Q).
\end{equation}

\subsection{Átmeneti Jacobian: $F_k$}
Az EKF-hez szükséges a mozgásmodell Jacobianja az állapot szerint:
\begin{equation}
\label{eq:jacobian_F_def}
F_k = \left.\frac{\partial f(x,u)}{\partial x}\right|_{x=x_k,u=u_k}.
\end{equation}
A (\ref{eq:kinematics}) modell esetén:
\begin{equation}
\label{eq:jacobian_F}
F_k =
\begin{bmatrix}
1 & 0 & -v(k)\,dt\,\sin(\psi(k))\\
0 & 1 & \phantom{-}v(k)\,dt\,\cos(\psi(k))\\
0 & 0 & 1
\end{bmatrix}.
\end{equation}

\subsection{Mérési modell: távolságmérések anchorokhoz}
Legyen $M$ darab rögzített anchor ismert pozícióval:
\begin{equation}
\label{eq:anchor}
a_i =
\begin{bmatrix}
a_{x,i}\\
a_{y,i}
\end{bmatrix},
\quad i=1,\dots,M.
\end{equation}
A mérési vektor a robot és az anchorok közötti távolságok:
\begin{equation}
\label{eq:measurement_vector}
z_k =
\begin{bmatrix}
z_1(k)\\
\vdots\\
z_M(k)
\end{bmatrix}.
\end{equation}
Az $i$-edik mérés modellje:
\begin{equation}
\label{eq:range_model}
z_i(k) = h_i(x_k) + v_i(k)
       = \sqrt{(a_{x,i}-p_x(k))^2 + (a_{y,i}-p_y(k))^2} + v_i(k),
\end{equation}
ahol $v_i(k)$ a mérési zaj. Vektorosan:
\begin{equation}
\label{eq:measurement_model}
z_k = h(x_k) + v_k, \qquad v_k \sim \mathcal{N}(0,R).
\end{equation}

\subsection{Mérési Jacobian: $H_k$}
A mérési modell Jacobianja:
\begin{equation}
\label{eq:jacobian_H_def}
H_k = \left.\frac{\partial h(x)}{\partial x}\right|_{x=x_k}.
\end{equation}
Mivel a mérés csak a pozíciótól függ, az orientáció szerint a derivált zérus.
Az $i$-edik sor (az $i$-edik anchorhoz tartozó deriváltak) alakja:
\begin{equation}
\label{eq:jacobian_H_row}
\frac{\partial h_i}{\partial p_x} = \frac{p_x-a_{x,i}}{\rho_i},\qquad
\frac{\partial h_i}{\partial p_y} = \frac{p_y-a_{y,i}}{\rho_i},\qquad
\frac{\partial h_i}{\partial \psi} = 0,
\end{equation}
ahol
\begin{equation}
\label{eq:rho}
\rho_i = \sqrt{(p_x-a_{x,i})^2 + (p_y-a_{y,i})^2}.
\end{equation}
Így a teljes Jacobian mátrix:
\begin{equation}
\label{eq:jacobian_H}
H_k =
\begin{bmatrix}
\frac{p_x-a_{x,1}}{\rho_1} & \frac{p_y-a_{y,1}}{\rho_1} & 0\\
\vdots & \vdots & \vdots\\
\frac{p_x-a_{x,M}}{\rho_M} & \frac{p_y-a_{y,M}}{\rho_M} & 0
\end{bmatrix}.
\end{equation}

\subsection{Zajmodellek és outlier modell}
A folyamatzaj Gauss eloszlású \cite{Kalman1960}:
\begin{equation}
\label{eq:process_noise_dist}
w_k \sim \mathcal{N}(0,Q).
\end{equation}
A mérési zaj baseline esetben szintén Gauss:
\begin{equation}
\label{eq:measurement_noise}
v_k \sim \mathcal{N}(0,R), \qquad R=\sigma_r^2 I.
\end{equation}
A szimulációban azonban ritkán \textit{outlier} is előfordul: egy véletlen anchor mérésére nagyobb szórású zaj adódik.
Ezt a jelenséget egy kevert (mixture) zajmodell szemlélteti \cite{Chhabra2021}:
\begin{equation}
\label{eq:outlier_model}
v_i(k) \sim
\begin{cases}
\mathcal{N}(0,\sigma_r^2), & \text{normál esetben},\\
\mathcal{N}(0,\sigma_{\text{out}}^2), & \text{outlier esetén},
\end{cases}
\end{equation}
ahol az outlier bekövetkezésének valószínűsége $p_{\text{out}}$.
A fentiek a későbbi fejezetekben motiválják az adaptív mérési megbízhatóság-becslést (B) és a gráf-alapú reziduál simítást (C).
