\section{Bevezetés és célkitűzés}

\subsection{Motiváció: lokalizáció fix anchor pontokkal}
Számos robotikai és ipari alkalmazásban a jármű vagy robot pozícióját egy ismert, rögzített pontokból (anchorokból) álló hálózat segítségével becsüljük \cite{Chai2020}.
A tipikus mérés az anchor--robot távolság (range), amelyet például rádiós (UWB), ultrahangos vagy egyéb időmérésen alapuló technológiák szolgáltatnak.
Az anchorok pozíciói ismertek és időben állandók, ezért a lokalizáció feladata lényegében a robot állapotának (pozíció és orientáció) online becslése zajos távolságmérésekből.

A dolgozatban egy egyszerű, de jól reprodukálható szimulációs környezetet használunk:
$M=5$ darab rögzített anchor, és egy mozgó robot, amelynek állapota
\begin{equation}
\label{eq:intro_state}
x_k = \begin{bmatrix} p_x(k) \\ p_y(k) \\ \psi(k) \end{bmatrix},
\end{equation}
ahol $p_x, p_y$ a pozíció, $\psi$ pedig a yaw szög. A robot mozgását diszkrét idejű kinematikai modell írja le, a mérések pedig az anchorokhoz tartozó távolságok.

\subsection{Probléma: zajos mérések és outlierek hatása az EKF-re}
A gyakorlatban a range mérések zajosak, továbbá időnként hibás értékek (outlierek) is megjelennek (pl.\ NLOS terjedés, multipath, szenzorhibák).
Ezek a hibák az állapotbecslőben -- különösen a frissítési lépésben -- jelentős eltéréseket okozhatnak:
egy-egy rossz mérés túl nagy korrekciót eredményezhet, ami a pálya ``elugrásához'' és a hibák felhalmozódásához vezet.

Az \textit{Extended Kalman Filter} (EKF) alapvetően Gauss zajfeltevésre épít \cite{Kalman1960}:
a mérési zajt az $R$ kovarianciamátrix modellezi. Ha $R$ alulbecsüli a valós hibát (pl.\ outlier esetén), akkor a szűrő túlzottan megbízik a mérésben, és instabilabbá válhat.
Ez motiválja a mérési megbízhatóság \textit{online} becslését és a mérési információ adaptív kezelését.

\subsection{Cél: mérési megbízhatóság becslése és gráf-alapú reziduál simítás}
A munka első célja egy gyors, ``kézzelfogható'' baseline felépítése és tesztelése:
\begin{itemize}[noitemsep]
    \item EKF lokalizáció fix anchorokkal, rögzített $R$ mátrixszal.
    \item Heurisztikus, reziduál-alapú mérési súlyozás: az egyes anchor mérések megbízhatóságát a reziduálok statisztikájából becsüljük.
\end{itemize}

A második cél egy gráfalapú jelfeldolgozási (GSP) elem beépítése ugyanebbe a keretbe \cite{Ortega2022, Shuman2013}.
A kulcsötlet, hogy egy adott időpillanatban az anchorokhoz tartozó reziduálok egy $r(k)\in\mathbb{R}^M$ vektort alkotnak:
\begin{equation}
\label{eq:intro_residual}
r(k) = z(k) - \hat z(k),
\end{equation}
ahol $z(k)$ a mért range vektor, $\hat z(k)$ pedig a predikált mérés.
Ha a reziduálok között strukturált kapcsolat van (pl.\ több anchor egyszerre romlik el NLOS miatt), akkor ez a struktúra gráfként modellezhető.
Ennek megfelelően:
\begin{itemize}[noitemsep]
    \item \textbf{Gráf csomópontok:} anchorok
    \item \textbf{Jel a gráfon:} reziduál $r(k)$
    \item \textbf{Élsúly:} a csomópontok közötti hasonlóság a reziduál-idősor alapján
\end{itemize}

A cél, hogy a tanult (vagy heurisztikusan felépített) gráfon egy egyszerű aluláteresztő (low-pass) simítást végezzünk a reziduálra, majd a szűrt reziduált használjuk fel az EKF frissítésben \cite{Dong2019}.
Ez várhatóan csökkenti az outlierek hatását és stabilabb becslést ad.

\subsection{Hozzájárulás: három pipeline összehasonlítása (A/B/C)}
A megvalósítás és értékelés során három különböző pipeline-t hasonlítunk össze ugyanazon szimulációs környezetben:

\begin{itemize}
    \item \textbf{(A) Baseline EKF fix mérési kovarianciával:} $R=\sigma_r^2 I$.
    \item \textbf{(B) Heurisztikus EKF adaptív mérési kovarianciával:} anchoronkénti súly $w_i(k)$ reziduál-energia alapján, majd $R_i(k)$ ennek függvényében.
    \item \textbf{(C) EKF + gráf-alapú reziduál simítás:} a reziduálokból tanult Laplacián segítségével a reziduál aluláteresztő szűrése, és frissítés a simított reziduállal.
\end{itemize}

A pipeline-ok összehasonlítását a pozíció becslési hiba (RMSE) és a hibagörbék alapján végezzük.
A szimuláció eredményeit automatikusan exportáljuk, és a generált ábrákat a dokumentumban felhasználjuk.
