\section{Szimulációs eredmények}

\subsection{Szimulációs beállítások}
A három módszer összehasonlítását egy kontrollált, reprodukálható szimulációs környezetben végeztük \cite{Chai2020}.
A szimuláció paraméterei az \ref{tab:simparams}. táblázatban láthatók.

\begin{table}[!ht]
\centering
\begin{tabular}{l l}
\hline
\textbf{Paraméter} & \textbf{Érték} \\
\hline
Anchorok száma ($M$) & 5 \\
Szimulációs lépések ($T$) & 350 \\
Mintavételi idő ($dt$) & 0.1 s \\
Mérési zaj szórása ($\sigma_r$) & 0.18 m \\
Outlier valószínűség ($p_{\text{out}}$) & 0.06 \\
Outlier szórás ($\sigma_{\text{out}}$) & 1.2 m \\
\hline
\end{tabular}
\caption{Szimulációs paraméterek}
\label{tab:simparams}
\end{table}

A robot mozgása egyszerű kinematikai modellel történik, konstans $v=0.85$ m/s sebesség és $\omega=0.22$ rad/s szögsebesség bemenetekkel, így egy folyamatosan kanyarodó pályát kapunk.
A szimulációt rögzített véletlenmaggal futtattuk a reprodukálhatóság érdekében.

\subsection{Összehasonlított módszerek}
Azonos mérési adatsoron három pipeline fut párhuzamosan:
\begin{itemize}[noitemsep]
    \item \textbf{(A) EKF fix mérési kovarianciával:} $R=\sigma_r^2 I$ -- ez a baseline referencia \cite{Kalman1960, Grewal2001}.
    \item \textbf{(B) EKF heurisztikus adaptív $R(k)$-val:} anchoronkénti súly $w_i(k)$ a reziduál-energiából, $\beta=2.0$, ablakméret 15 \cite{Chhabra2021}.
    \item \textbf{(C) EKF + Dong-típusú Laplacián tanulás + reziduál simítás:} $\gamma=1.0$, ablakméret 20, tanulás 10 lépésenként \cite{Dong2019}.
\end{itemize}

\subsection{RMSE összehasonlítás}
A három módszer teljesítményét a pozíció RMSE mutatóval értékeltük. A kapott eredményeket a \ref{tab:rmse}. táblázat foglalja össze.

\begin{table}[!ht]
\centering
\begin{tabular}{l c c}
\hline
\textbf{Módszer} & \textbf{RMSE [m]} & \textbf{Javulás A-hoz képest} \\
\hline
(A) Fix R & 0.0801 & -- \\
(B) Adaptív R & 0.0738 & 7.9\% \\
(C) GSP simítás & 0.0706 & 11.9\% \\
\hline
\end{tabular}
\caption{RMSE eredmények a három módszerre}
\label{tab:rmse}
\end{table}

A baseline (A) módszerhez viszonyítva mind a (B), mind a (C) módszer jelentős javulást ért el. A legjobb eredményt a (C) pipeline adta, amely közel 12\%-kal csökkentette az átlagos pozícióhibát. Ez összhangban van a GSP-alapú megközelítések előnyeivel, amelyeket \cite{Ortega2022} és \cite{Shuman2013} is kiemelnek.

\subsection{Trajektóriák összehasonlítása}
Az \ref{fig:traj}. ábra a valódi robot pályát és a három módszer becslését mutatja. Látható, hogy a baseline (A) módszer időnként jobban eltér a valódi pályától, különösen ott, ahol outlier mérések zavarják a becslést. A (B) és (C) módszerek közelebb maradnak a valódi trajektóriához.

\begin{figure}[!ht]
    \centering
    \includegraphics[width=0.9\linewidth]{traj.png}
    \caption{Trajektóriák összehasonlítása: a fekete vonal a valódi pálya, a szaggatott vonalak a három módszer becslései. Az anchorok piros X-szel jelölve.}
    \label{fig:traj}
\end{figure}

\subsection{Pozícióhiba időbeli alakulása}
Az \ref{fig:error}. ábra a pillanatnyi pozícióhibát mutatja az idő függvényében mindhárom módszerre. A baseline (A) hibája időnként nagyobb csúcsokat mutat -- ezek jellemzően az outlier mérésekkel egybeesnek. A (B) és (C) módszerek ezeket a csúcsokat hatékonyan tompítják.

\begin{figure}[!ht]
    \centering
    \includegraphics[width=0.95\linewidth]{error.png}
    \caption{Pozícióhiba időben: $e(k)=\|\hat p(k)-p(k)\|_2$ a három pipeline esetén. A (B) és (C) módszerek csökkentik a hibaugrásokat.}
    \label{fig:error}
\end{figure}

\subsection{Heurisztikus súlyok alakulása (B módszer)}
Az \ref{fig:weights}. ábra a (B) pipeline heurisztikus súlyait mutatja anchoronként. Amikor egy anchor mérése tartósan zajos vagy outlier-érintett, a súlya csökken, így az EKF automatikusan kevésbé bízik az adott mérésben. A súlyok 0 és 1 között mozognak, ahol 1 a teljes megbízhatóságot jelenti.

\begin{figure}[!ht]
    \centering
    \includegraphics[width=0.95\linewidth]{weights.png}
    \caption{A (B) módszer heurisztikus súlyai időben. Alacsonyabb súly azt jelzi, hogy az anchor mérése kevésbé megbízható.}
    \label{fig:weights}
\end{figure}

\subsection{Reziduál simítás hatása (C módszer)}
Az \ref{fig:residnorm}. ábra a (C) pipeline-ben a nyers és a gráf-alapú simított reziduál normáját hasonlítja össze. A simítás jellemzően csökkenti a reziduál nagyságát, különösen a kiugró értékek esetén, ami stabilabb EKF frissítést eredményez. Ez a low-pass szűrési tulajdonság közvetlenül következik a GSP elméletéből \cite{Shuman2013, Ortega2018GSP}.

\begin{figure}[!ht]
    \centering
    \includegraphics[width=0.95\linewidth]{residual_norm.png}
    \caption{Reziduál norma: nyers $\|r\|$ és GSP-simított $\|\tilde r\|$. A simítás csökkenti az outlierek hatását.}
    \label{fig:residnorm}
\end{figure}

\subsection{Paraméterérzékenység}
A módszerek teljesítménye függ a hangolási paraméterektől. Egy szélesebb körű paraméter-sweep vizsgálat alapján a következő megfigyeléseket tehetjük:

\begin{itemize}[noitemsep]
    \item \textbf{Outlier valószínűség hatása:} Minél több az outlier, annál nagyobb a (B) és (C) módszerek előnye a baseline-hoz képest.
    \item \textbf{$\beta$ paraméter (B módszer):} A $\beta=2.0$ érték körül optimális. Túl nagy $\beta$ (pl.\ 4.0) esetén a módszer túlérzékeny lesz és instabillá válhat.
    \item \textbf{$\gamma$ paraméter (C módszer):} Nagyobb $\gamma$ erősebb simítást jelent. A $\gamma=1.0$--$2.0$ tartomány általában jó eredményt ad \cite{Dong2019}.
    \item \textbf{Zajszint hatása:} Alacsony zajszintnél ($\sigma_r=0.1$ m) a (C) módszer egyértelműen jobb. Magas zajszintnél ($\sigma_r=0.3$ m) a különbségek csökkennek.
\end{itemize}
